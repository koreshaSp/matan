\section*{Лекция 6 (24.03.2022)}


В прошлой серии:

\[
    U : X \to Y \text{ --- линейное непрерывное}
\]

\[
    \norm{U} = \inf \{ C : \forall x \in X \ \norm{Ux} \leqslant C \norm{x} \}
\]

\[
    \forall x : \norm{Ux} \leqslant \norm{U} \cdot \norm{x} \hence \norm{U} \geqslant \frac{\norm{Ux}}{\norm{x}}
\]

\[
    \norm{U} = \sup_{x \not= 0}\left(\frac{\norm{Ux}}{\norm{x}}\right)
\]
\[
    \norm{U} = \sup_{\norm{x}= 1 }{\frac{\norm{Ux}}{\norm{x}}}, \text{ но не максимум, т.к. сфера не обязательно компактна}
\]
\\
\exercise  $\norm{U} = \sup_{\norm{x} < 1}{\norm{Ux}} = \sup_{\norm{x} \leqslant 1} \norm{Ux}$
\\
\begin{remark}
    $
        U : X_1 \times X_2 ... \times X_m \to Y
    $ - полилинейное отображение

    \[
        \norm{U} = \inf \{ C: \forall x_1 \in X_1, ..., \forall x_n \in X_m \ \norm{U(x_1, ..., x_n)} \leqslant C \norm{x_1} ... \norm{x_m}\}
    \]
    
    \[
        \hence \forall x_1, ..., x_m \norm{U(x_1, ... , x_m)} \leqslant \norm{U} \cdot \norm{x_1} ... \norm{x_m}
    \]

    \[
        \norm{U} = sup_{x_1,..,x_n} \frac{\norm{U(x_1, x_2, ... x_n)}}{\norm{x_1} ... \norm{x_n}}
    \]

\end{remark}
\newpage
\begin{example}
    \[
        D: C^1[0, 1] \to C[0, 1]; (Df)(x) = f'(x)
    \]

    Норма в $C^1[0, 1]$:

    \begin{enumerate}
        \item $
            \phi_1(f) = max_{x \in [0, 1]} \abs{f(x)}  
        $ 
        \item $
            \phi_2(f) = max_{x \in [0, 1]} \abs{f'(x)}  
        $ --- не норма (т.к. обнуляется на константах)
        \item $
            \phi_3(f) = \phi_1(f) + \phi_2(f)
        $
    \end{enumerate}

    $(C^1[0, 1], \phi_1)$ --- не полное пространство (берем непрерывную не дифференцируемую функцию и приближаем многочленом, например $\abs{x} ^ {1 - \varepsilon} \to \abs{x} $)

    \exercise  $(C^1[0, 1], \phi_3)$ --- полное пространство


    \[
        \norm{Df}_{C[0, 1]} \leqslant C \norm{f}_{C^1 [0,1]}
    \]

    $c = 1$ годится

    \[
        \norm{f'}_{C[0, 1]} \leqslant (\norm{f'}_{C[0,1]} + \norm{f}_{C[0, 1]})
    \]

    \[
        f_k(x) = sin(kx);
        \norm{f'_k}_c = k;
        \norm{f_k} = 1
    \]
\end{example}

\subsection{О пространстве линейных непрерывных отображений}
\begin{theorem}
    $X, Y$ - линейные нормированные пространства над одним полем скаляров $(\R, \mathbb{C})$ \\ $L(X, Y) = \{ U : X \to Y, U \text{--- линейно непрерывное} \}$ --- линейное множество (линейное пространство)

    \begin{enumerate}
        \item $\norm{U_1 + U_2} \leqslant \norm{U_1} + \norm{U_2}$
        \item $\norm{\lambda U} = \abs{\lambda} \cdot \norm{U}$
        \item $\norm{U} = 0 \same U = 0$
        \item $U \in L(X, Y), V \in L(Y, Z) \hence VU \in L(X, Z)$ и $\norm{VU} \leqslant \norm{V} \cdot \norm{U}$
    \end{enumerate}
\end{theorem}



\begin{proof}
   \begin{enumerate}
       \item 
       \[
           \norm{U_1 + U_2} = \sup_{\norm{x} = 1}\norm{U_1x + U_2x} \le
           \sup_{\norm{x}=1}(\norm{U_1 x} + \norm{U_2 x}) \le
           \sup_{\norm{x}=1}\norm{U_1 x} + \sup_{\norm{x} = 1}\norm{U_2 x} = \norm{U_1} + \norm{U_2}
       \]
       \item \[
           \norm{\lambda u} = \sup_{\norm{x} = 1} {\norm{\lambda Ux}} = \abs{\lambda} \sup_{\norm{x} = 1}{\norm{Ux}} = \abs{\lambda} \cdot \norm{U}
       \]

       \item Без комментариев
       
       \item \[
           \norm{VUx} \leqslant \norm{V} \cdot \norm{Ux} \leqslant \underbrace{\norm{V} \cdot \norm{U} }_{ = C}  \cdot \norm{x} \hence \norm{V \cdot U} \leqslant \norm{V} \cdot \norm{U}
       \]
   \end{enumerate} 
        

\end{proof}



\begin{theorem}
    $L(X, Y)$ --- полное, если $Y$ полное
\end{theorem}


\begin{proof}
    Выберем произвольную фундаментальную последовательность $\{U_k \in L(X, Y)\}$ и докажем, что она сходится к некоторому $U$,
    тем самым доказав, что $L(X,Y)$~--- полное пространство.

    Будем делать это следующим образом: определим $U(x) = \lim_{n\to \infty}(U_n x)$ и докажем для него следующие свойства:
    \begin{enumerate}
        \item
            $\norm{U_n - U} \underset{n\to\infty}{\to} 0$
        \item 
            Линейность $U$.
        \item
            Непрерывность $U$, что в совокупности с линейностью даст $U\in L(X,Y)$.
    \end{enumerate}
    \begin{enumerate}
        \item
        По фундаментальности $\{U_k\}$ можно выписать следующее:
        \[
            \begin{gathered}
                \forall \varepsilon > 0, \exists N\colon \forall n, m > N, \norm{U_n - U_m} < \varepsilon\\
                \Downarrow\\
                \forall \varepsilon > 0, \exists N\colon \forall n, m > N, \forall x\in X, \norm{U_n - U_m}\norm{x} < \norm{x}\varepsilon\ora\\
                \left[\norm{Ux}\le \norm{U}\norm{x},\forall x\in X, U\in L(X,Y)\right]\\
                \ora\forall \varepsilon > 0, \exists N\colon \forall n, m > N, \forall x\in X, \norm{U_nx - U_mx} < \norm{x}\varepsilon \ora\\
                \left[\text{при $n\to\infty$ получаем, что $Ux = U_nx$ просто по определению $U$}\right]\\
                \ora \forall \varepsilon > 0, \exists N\colon \forall m > N, \norm{U - U_m} < \varepsilon \Rightarrow
                \norm{U-U_m}\underset{m\to\infty}{\to}0
            \end{gathered}
        \] 
    \item 
        \[
        \begin{gathered}
            U(x_1 + x_2) = \lim_{n\to\infty}(U_n(x_1 + x_2)) = \lim_{n\to\infty}(U_nx_1 + U_nx_2) =
            \lim_{n\to\infty}(U_nx_1) + \lim_{n\to\infty}(U_nx_2) = Ux_1 + Ux_2
        \end{gathered}
        \] 
    \item
        По фундаментальности $\{U_k\}$ можно выписать следующее:
        \[
        \begin{gathered}
            \forall \varepsilon > 0, \exists N\colon \forall n, m > N, \norm{U_n - U_m} < \varepsilon\\
            \Downarrow\\
            \forall \varepsilon > 0, \exists N\colon \forall n, m > N, \forall x\in X, \norm{U_n - U_m}\norm{x} < \norm{x}\varepsilon\ora\\
            \left[\norm{Ux}\le \norm{U}\norm{x},\forall x\in X, U\in L(X,Y)\right]\\
            \ora\forall \varepsilon > 0, \exists N\colon \forall n, m > N, \forall x\in X, \norm{U_nx - U_mx} < \norm{x}\varepsilon \ora\\
            [\text{зафиксируем $\varepsilon = 1$ и устремим $n\to\infty$, получив $U = U_n$ по определению}]\\
            \ora\exists N\colon \forall m > N, \forall x\in X, \norm{Ux - U_mx} < \norm{x}
        \end{gathered}
        \] 
        Из последнего выражения следует непрерывность функции $U - U_m$ просто из-за того, что 
        $\norm{U - U_m x} \le C\norm{x}, \forall x\in X$(см теорема об эквивалентных условиях непрерывности).
        Тогда $U\in L(X,Y)$ так как $U - U_m\in L(X,Y), U_m\in L(X,Y)$.
    \end{enumerate}
\end{proof}


\begin{theorem}
    (штрих) $L(X_1, ..., X_n; Y)$ --- полное, если $Y$ --- полное, где $L$ --- полилинейные, непрерывные отображения из $X_1 \times ... \times X_n \to Y$
\end{theorem}

\begin{theorem}
    $X_1, ... X_s, X_{s + 1},...,X_k; Y$ - линейные нормированные пространства

    $A = L(X_1, ..., X_s; L(X_{s + 1}, ..., X_k; Y)) \simeq L(X_1, ..., X_k; Y) = B$,
    где $\simeq$~--- изометрический изоморфизм(биекция, сохр. линейную струкутуру и норму)
\end{theorem}

\begin{proof}
    $u \in A, x_1 \in X_1, ..., x_s \in X_s \hence U(x_1, ..., x_s) \in L(x_{s + 1}, ..., x_k; Y)$

    $x_{s + 1} \in X_{s + 1}, ..., x_k \in X_k \hence U(x_1,..., x_s)(x_{s + 1}, ..., x_{k}) \in Y$
    
    $U^{\sim} \in B;  U^{\sim} (x_1, ... x_k) := U(x_1, ..., x_s)(x_{s + 1}, ..., x_k)$ --- полилинейное отображение, непрерывное

    $\norm{U^\sim} = \sup_{x_1,..x_k} \cfrac{\norm{U(x_1, ... x_s)(x_{s+1}, ..., x_k)}}{\norm{x_1} ... \norm{x_k}} = \sup_{x_1, ..., x_s} \cfrac{ \sup_{x_{s+1}, ..., x_k} \cfrac{\norm{U(x_1, ..., x_s)(x_{s + 1}, ... , x_k)}}{\norm{x_{s + 1}} ... \norm{x_k}}}{\norm{x_1} ... \norm{x_s}} =$

    $ = \sup_{x_1,..., x_s}{\cfrac{\norm{U(x_1,...,x_s)}}{\norm{x_1} ... \norm{x_s}}} = \norm{U}$
\end{proof}


\section{Дифференциал}
\subsection{Основные свойства}
\begin{definition}
    $U \subset X$--- открытое, $f : U \to Y$, $X, Y$ - линейные полные нормированные пространства, $x_0 \in U$
    

    $f$ дифференцируемо в $(\cdot)\, x_0$, если $f(x_0 + h) = f(x_0) + Ah + \underset{ = \alpha(h)}{o(h)}, h \to 0$, где $A \in L(X, Y)$, \\ и $\cfrac{\norm{\alpha(h)}}{\norm{h}} \to 0, h \to 0$

    $A$ --- дифференциал $f$ в $(\cdot) x_0$

    $A = df(x_0) = d_{x_0}f = D_{x_0}f = f'(x_0)$ (не производная !!!)
\end{definition}


\newpage

\begin{remark}

\quad

\begin{enumerate}
    \item  Если $X$ --- конечномерно, то из линейности следует непр., т.к. $\exists \, C = max_{\norm{x} = 1} \norm{Ux} < \infty$
    \item \begin{definition}
        в $\infty$ мерном пространстве в лит-ре называется дифференцируемость по Фреше
    \end{definition}
\end{enumerate}
   
\end{remark}

\begin{properties}{}
    \item $f$ -- дифф в $(\cdot) x_0 \hence f $ --- непр. в $(\cdot) x_0$
    \begin{proof}
        $f(x_0 + h) = f(x_0) + \underbrace{df(x_0)h + o(h)}_{\xrightarrow{h \to 0}0}$
    \end{proof} 
    \item дифференциал единственен
    \begin{proof}
        $f(x_0 + h) = f(x_0) + A_1(h) + o(h) = f(x_0) + A_2(h) + o(h) \hence (A_1 - A_2)(h) = o(h)$

        $\cfrac{\norm{(A_1 - A_2)h}}{\norm{h}} \xrightarrow{h \to 0} 0$
    
        Берём $h_0 - fix, h = t \cdot h_0, t \to 0, t - $ скаляр

        $\cfrac{\norm{(A_1 - A_2)(t \cdot h_0)}}{\norm{t \cdot h_0}}  
        = \cfrac{\norm{(A_1 - A_2)(h_0)}}{\norm{h_0}} \xrightarrow{t \to 0} 0 \hence 
        = 0$, т.к. не зависит от $t$ и стремится к $0$ 
    \end{proof}

    \item (линейность) $f, g$ --- дифф. в $(\cdot) x_0 \hence \alpha f + \beta g $ --- дифф  в $(\cdot) x_0 $
    
    $d(\alpha f + \beta g)(x_0) = \alpha df(x_0) + \beta dg(x_0)$

    \item (дифференциал композиции --- правило цепочки)
    
    $x_0 \in U \subset X, f(x_0) \in V \subset Y, Z$

    $f : U \to Y, g : V \to Z$

    $f$ --- дифф. в $(\cdot)\, x_0$, 
    $g$ --- дифф. в $(\cdot)\, f(x_0)$

    $\hence g \circ f $ дифф в $(\cdot)\, x_0$ и $\underbrace{d(g \circ f) (x_0)}_{\in L(X, Z)}  = \underbrace{dg(f(x_0))}_{\in L(Y, Z)} \circ \underbrace{df(x_0)}_{\in L(X, Y)}$
\newpage
    \begin{proof}
        $g(f(x_0 + h)) = g(f(x_0)) + dg(f(x_0))(f(x_0 + h) - f(x_0)) + o(f(x_0 + h) - f(x_0)) = $ 
        
        $= g(f(x_0)) + dg(f(x_0))(df(x_0)h + o(h)) + o(f(x_0 + h) - f(x_0)) = $


        $= g(f(x_0)) + \underbrace{dg(f(x_0)) df(x_0)}_{\in L(X, Z)} h + \underbrace{dg(f(x_0))(o(h)) + o(f(x_0 + h) - f(x_0))}_{?\ceq o(h)}$

        $\norm{dg(f(x_0))} \leqslant \norm{dg(f(x_0))} \cdot \underbrace{\norm{o(h)}}_{=o(h)}$

        $\norm{f(x_0 + h) - f(x_0)} \leqslant C \norm{h} \hence \ceq o(h)$
    \end{proof}

    \item (дифференциал обратного отображения)
    
    $U \subset X, V \subset Y, f : U \to V $ --- биекция, $f $ --- непр. в $(\cdot)\, x_0$, $f^{-1}$ --- непр. в $(\cdot)\, f(x_0) = y_0$ 

    $f$ дифф в $(\cdot) \, x_0$

    $\exists (df(x_0))^{-1} \in L(Y, X)$

    $\hence f^{-1}$ дифф в $(\cdot)\, y_0$ и $df^{-1}(y_0) = (df(x_0)) ^ {-1}$
\end{properties}
