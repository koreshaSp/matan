\section*{Лекция 2}

\begin{example}

 \[
     W_n = \int_0^{\pi / 2}{(\sin{x})^n} dx = \int_0^{\pi / 2}{(\cos{x})^n}dx \text{ (т.к. можно сделать замену $y = \pi / 2 - x$)} 
\]

\[
    W_n = \int_0^{\pi / 2}\cos^n{x}dx = \int_0^{\pi / 2}\cos^{n - 1}{x} d(\sin{x}) = \cos^{n - 1}{x} \cdot \sin{x} \bigg|_{0}^{\pi/2} - \int_{0}^{\pi / 2}\sin{x} d(\cos^{n - 1}{x}) = 
\]

\[
    = 0 \text{ (при $n \geqslant 2$)} - \int_0^{\pi / 2} \sin{x} \cdot (n - 1) \cos^{n - 2}{x}(-\sin{x}) dx = (n - 1)\int_{0}^{\pi / 2}(1 - \cos^2{x}) \cos^{n - 2}{x} dx = 
\]

\[
    = (n - 1)(W_{n - 2} - W_{n})
\]

Итого получаем рекуректную формулу:

\[
    W_n = \frac{n - 1}{n} \cdot W_{n - 2}, \, W_0 = \pi / 2, \, W_1 = 1
\]

Тогда итоговая формула:

\[
    W_{2k} = \frac{2k - 1}{2k} \cdot \frac{2k - 3}{2k - 2} \cdot ... \cdot \frac{1}{2} \cdot W_0 = \frac{(2k - 1)!!}{(2k)!!} W_0 = \frac{(2k - 1)!!}{(2k)!!} \pi / 2
\]

\[
    W_{2k + 1} = \frac{2k}{2k + 1} \cdot \frac{2k - 2}{2k - 1} \cdot ... \cdot \frac{2}{3} \cdot W_1 = \frac{(2k)!!}{(2k + 1)!!} W_1 = \frac{(2k)!!}{(2k + 1)!!} 
\]
\\

\textbf{(Офф-топ)}: Можно заметить, что $(2k)!! = 2 ^ k \cdot k!$ и $(2k + 1)!! = \frac{(2k + 1)!}{(2k)!!} = \frac{(2k + 1)!!}{2^k \cdot k!}$
\\
\newpage
Также, стоит отметить, что $W_n \geqslant W_{n + 1} \geqslant W_{n + 2} \geqslant ...$, т.к. $\sin^n{x} \leqslant \sin^{n - 1}{x}$, то есть интегрируем меньшую функцию, на одном и том же промежутке.
Также, заметим, что это на самом деле есть неплохое приблежение числа $\pi$.\\

\underline{\textit{Формула Валлиса}}

\[
    W_{2k + 2} \leqslant W_{2k + 1} \leqslant W_{2k}
\]

\[
    \pi / 2 \cdot \frac{(2k + 1)!!}{(2k + 2)!!} \leqslant \frac{(2k)!!}{(2k + 1)!!} \leqslant \pi / 2 \cdot \frac{(2k - 1)!!}{(2k)!!}
\]

\[
    \pi / 2 \cdot \frac{2k + 1}{2k + 2} \leqslant \frac{((2k)!!)^2}{(2k + 1)((2k - 1)!!) ^ 2} \leqslant \frac{\pi}{2}
\]

\[
    \Downarrow
\]
\[
    \lim_{k \to \infty} \frac{(2k!!)^2}{(2k + 1)((2k - 1)!!) ^ 2} = \pi / 2 \Rightarrow  \lim_{k \to \infty} \frac{2k!!}{\sqrt{2k + 1}(2k - 1)!!} = \sqrt{\pi / 2}
\]
\\

\follow 

\[
    \sum_{k = 0}^{2n} C_{2n}^{k} = 4 ^ n, \,
    C_{2n}^n = \frac{(2n)!}{(n!)^2}
\]

\[
    \frac{(2k)!!}{(2k - 1)!!} = \frac{(2^k \cdot k!)^2}{(2k)!} = \frac{4 ^ k \cdot (k!)^2}{(2k)!}
\]

По формуле Валлиса:

\[
    \frac{1}{\sqrt{2k + 1}} \cdot \frac{4 ^ k}{C_{2k}^k} \to \sqrt{\pi / 2} \Leftrightarrow C_{2k}^k \sim \sqrt{\frac{2}{\pi}} \cdot \frac{4 ^ k}{\sqrt{2k + 1}}
\]

\end{example}



\newpage

\section{Формула Тейлора с остатком в интегральной форме}

\quad

$f(x) = T_{n, x_0} f(x) + R_{n, x_0} f(x)$, где $T_{n, x_0} f(x) = f(x_0) + f'(x_0)(x - x_0) + ... + \frac{f^{(n)}(x_0)}{n!}(x - x_0)^n$

\begin{namedtheorem}{остаток в интегральной форме}
    
    Если \, $f \in C^{n + 1} \langle a, b \rangle$,  \,\, $x, x_0 \in \langle a, b \rangle$, то $
        R_{n, x_0}f(x) = \frac{1}{n!} \int_{x_0}^x f^{(n + 1)}(t)(x-t)^n dt
    $
\end{namedtheorem}

\begin{proof}

    \textbf{\large База:} 
    \[
        n = 0, \text{ то } f(x) = f(x_0) + \int_{x_0}^{x}f'(t)dt  \text{ --- формула Ньютона-Лейбница}
    \]

    \textbf{\large Переход:} 
    \[
        R_{n, x_0}f(x) = \frac{1}{n!} \int_{x_0}^{x}f^{(n+1)}(t)(x - t)^n dt = \frac{1}{n!} \cdot \int_{x_0}^x f^{(n + 1)}(t) d(-\frac{(x - t) ^ {n + 1}}{n + 1}) = 
    \]

    \[
        = - \frac{1}{n!} \cdot \frac{(x - t) ^ {n + 1}}{n + 1} f^{(n + 1)}(t) \bigg|_{x_0}^{x} + \frac{1}{n!} \int_{x_0}^{x} \frac{(x - t)^{n + 1}}{n + 1} d f^{(n + 1)}(t) = 
    \]

    \[
        = \frac{(x - x_0) ^ {n + 1}}{(n + 1)!} f^{n + 1}(x_0) + R_{n + 1, x_0} f(x)
    \]


\end{proof}


\begin{namedlemma}{чуть более хитрая теорема о среднем}
    
    $f, g \in C[a, b], g \geqslant 0 \Rightarrow \exists c \in (a, b) : \int_a^b f(x) g(x) dx = f(c) \int_a^b g(x) dx$
\end{namedlemma}

\newpage 
\begin{proof}
    \[
        m = \min_{[a, b]} f, \,\,\, M = \max_{[a, b]} f
    \]

    \[
        m g(x) \leqslant f(x) g(x) \leqslant M g(x)
    \]

    \[
        m \int_a^b g(x) dx \leqslant \int_a^b f(x) g(x) dx \leqslant M \int_a^b g(x) dx
    \]

    \[
        \frac{\int_a^b f(x) g(x) dx}{\int_a^b g(x) dx} \in [m, M] \Rightarrow \text{равно значению в некоторой промежуточной точке}
    \] 
\end{proof}


Берём остаток в интегральной форме:

\[
    R_{n, x_0} f(x) = \frac{1}{n!} \int_{x_0}^x f^{(n + 1)}(t) (x - t) ^ n dt = 
\]

\[
    \exists \theta : \frac{1}{n!} f^{(n + 1)} (\theta) \int_{x_0}^x (x - t) ^ n dt = \frac{1}{(n + 1)!} f^{(n + 1)}(\theta) \cdot (x - x_0) ^ {(n + 1)} \text{ --- остаток в форме Лагранжа}
\]

\quad

Некоторые студенты на экзамене стали жертвой следующего:

\[
    \ln{(1+x)} = x - \frac{x ^ 2}{2} + \frac{x ^ 3}{3} - ... + R_n
\]

\[
    R_n = \frac{n!}{(n + 1)!} \cdot \frac{(-1) ^ {n + 1}}{(1 + \theta) ^ n} \cdot x^{n + 1} = \frac{(-1) ^ {n + 1}}{n + 1} \cdot \frac{x ^ {n + 1}}{(1 + \theta) ^ n}
\]

\quad 

При каких $x$,\, $R_n \to 0$?

При $x \in (0, 1)$ утверждение очевидно, но что делать при $x \in (-1, 0)$?

\[
    R_n = \frac{1}{n!} \int_0^x ... \text{ To be continued (Федор Львович думал, что так можно, но не получилось :) )}
\]

\newpage 

\subsection{Приближенное вычисление интеграла}

\begin{definition}
    $[a, b], \tau $ --- дробление отрезка $[a, b]$ $\tau = \{ x_0, x_1, ..., x_n \}$, \\ где $a = x_0 < x_1 < ... < x_n = b$ \quad $|\tau| = \max_{k = 0, ..., n - 1}{|x_{k + 1} - x_k|}$ - мелкость (ранг) дробления

    $\theta = \{ t_1, ..., t_n\}$ -- оснащение дробления $\tau$, где $t_j \in [x_{j - 1}, x_j]$

    $(\tau, \theta) $ --- оснащенное дробление

    $f \in C[a, b]$ интегральная сумма $S_{\tau, \theta} (f) = \sum_{j = 0}^{n - 1} f(t_{j + 1}) (x_{j + 1} - x_j)$
\end{definition}



\quad


% Pattern Info
 
\tikzset{
pattern size/.store in=\mcSize, 
pattern size = 5pt,
pattern thickness/.store in=\mcThickness, 
pattern thickness = 0.3pt,
pattern radius/.store in=\mcRadius, 
pattern radius = 1pt}
\makeatletter
\pgfutil@ifundefined{pgf@pattern@name@_fhsu0lo8n}{
\pgfdeclarepatternformonly[\mcThickness,\mcSize]{_fhsu0lo8n}
{\pgfqpoint{0pt}{0pt}}
{\pgfpoint{\mcSize+\mcThickness}{\mcSize+\mcThickness}}
{\pgfpoint{\mcSize}{\mcSize}}
{
\pgfsetcolor{\tikz@pattern@color}
\pgfsetlinewidth{\mcThickness}
\pgfpathmoveto{\pgfqpoint{0pt}{0pt}}
\pgfpathlineto{\pgfpoint{\mcSize+\mcThickness}{\mcSize+\mcThickness}}
\pgfusepath{stroke}
}}
\makeatother
\tikzset{every picture/.style={line width=0.75pt}} %set default line width to 0.75pt        
\usetikzlibrary{patterns}
\begin{tikzpicture}[x=0.75pt,y=0.75pt,yscale=-1,xscale=1]
%uncomment if require: \path (0,300); %set diagram left start at 0, and has height of 300

%Shape: Axis 2D [id:dp02029949369434092] 
\draw  (132,233.4) -- (534.5,233.4)(172.25,66) -- (172.25,252) (527.5,228.4) -- (534.5,233.4) -- (527.5,238.4) (167.25,73) -- (172.25,66) -- (177.25,73)  ;
%Straight Lines [id:da7710677350311679] 
\draw    (223,222) -- (223,242) ;
%Straight Lines [id:da6636852601941825] 
\draw    (290,223) -- (290,243) ;
%Straight Lines [id:da500250663548284] 
\draw    (267,224) -- (267,244) ;
%Straight Lines [id:da377281395464541] 
\draw    (243,222) -- (243,242) ;
%Straight Lines [id:da8386061762887231] 
\draw    (311,223) -- (311,243) ;
%Straight Lines [id:da7815801182452056] 
\draw    (333,224) -- (333,244) ;
%Straight Lines [id:da8616890669141692] 
\draw    (345,225) -- (345,245) ;
%Shape: Free Drawing [id:dp5560591048641171] 
\draw  [color={rgb, 255:red, 0; green, 0; blue, 0 }  ,draw opacity=1 ][line width=0.75] [line join = round][line cap = round] (222,195) .. controls (243.45,162.83) and (256.46,175.18) .. (301,172) .. controls (305.12,171.71) and (312.5,172.5) .. (317,171) .. controls (327.51,167.5) and (340.44,159.22) .. (343,149) .. controls (343.86,145.54) and (344,120.8) .. (344,128) ;
%Straight Lines [id:da9019023552875143] 
\draw [color={rgb, 255:red, 208; green, 2; blue, 27 }  ,draw opacity=1 ] [dash pattern={on 4.5pt off 4.5pt}]  (223,193) -- (223,222) ;
%Straight Lines [id:da3323747111843628] 
\draw [color={rgb, 255:red, 208; green, 2; blue, 27 }  ,draw opacity=1 ] [dash pattern={on 4.5pt off 4.5pt}]  (243,175) -- (243,192) -- (243,222) ;
%Straight Lines [id:da04332372509475069] 
\draw [color={rgb, 255:red, 208; green, 2; blue, 27 }  ,draw opacity=1 ] [dash pattern={on 4.5pt off 4.5pt}]  (267,174) -- (267,224) ;
%Straight Lines [id:da18598253754944805] 
\draw [color={rgb, 255:red, 208; green, 2; blue, 27 }  ,draw opacity=1 ] [dash pattern={on 4.5pt off 4.5pt}]  (290,173) -- (290,223) ;
%Straight Lines [id:da9229530047708664] 
\draw [color={rgb, 255:red, 208; green, 2; blue, 27 }  ,draw opacity=1 ] [dash pattern={on 4.5pt off 4.5pt}]  (311,173) -- (311,223) ;
%Straight Lines [id:da27194960638393784] 
\draw [color={rgb, 255:red, 208; green, 2; blue, 27 }  ,draw opacity=1 ] [dash pattern={on 4.5pt off 4.5pt}]  (332,163) -- (333,224) ;
%Straight Lines [id:da8974921338138321] 
\draw [color={rgb, 255:red, 208; green, 2; blue, 27 }  ,draw opacity=1 ] [dash pattern={on 4.5pt off 4.5pt}]  (344,125) -- (345,225) ;
%Straight Lines [id:da25565245177699003] 
\draw [color={rgb, 255:red, 74; green, 144; blue, 226 }  ,draw opacity=1 ]   (223,193) -- (243,193) ;
%Straight Lines [id:da5135245657261043] 
\draw [color={rgb, 255:red, 74; green, 144; blue, 226 }  ,draw opacity=1 ]   (243,175) -- (267,174) ;
%Straight Lines [id:da40438608893718364] 
\draw [color={rgb, 255:red, 74; green, 144; blue, 226 }  ,draw opacity=1 ]   (267,174) -- (290,173) ;
%Straight Lines [id:da8267126008581839] 
\draw [color={rgb, 255:red, 74; green, 144; blue, 226 }  ,draw opacity=1 ]   (290,173) -- (311,173) ;
%Straight Lines [id:da6583698282228275] 
\draw [color={rgb, 255:red, 74; green, 144; blue, 226 }  ,draw opacity=1 ]   (311,173) -- (332,173) ;
%Straight Lines [id:da6130580258949248] 
\draw [color={rgb, 255:red, 74; green, 144; blue, 226 }  ,draw opacity=1 ]   (332,163) -- (346,163) ;
%Straight Lines [id:da9281438699801965] 
\draw    (276,227) -- (276,244) ;
%Shape: Rectangle [id:dp8370752348674357] 
\draw  [pattern=_fhsu0lo8n,pattern size=8.100000000000001pt,pattern thickness=0.75pt,pattern radius=0pt, pattern color={rgb, 255:red, 126; green, 211; blue, 33}] (267,174) -- (290,174) -- (290,233) -- (267,233) -- cycle ;

% Text Node
\draw (252,245.4) node [anchor=north west][inner sep=0.75pt]    {$x_{j}$};
% Text Node
\draw (291,246.4) node [anchor=north west][inner sep=0.75pt]    {$x_{j+1}$};
% Text Node
\draw (272,248.4) node [anchor=north west][inner sep=0.75pt]    {$t_{j}$};


\end{tikzpicture}



\begin{theorem}
    $f \in C[a, b] \Rightarrow S_{\tau, \theta}(f) \underset{|T| \to 0} {\to} \int_a^b f(x) dx$, \\ \\ т.е. $\forall \varepsilon > 0 \,\, \exists \delta > 0 : \forall \tau, \theta : | \tau | < \delta \Rightarrow \abs{ \int_a^b f dx - S_{\tau, \theta}(f)} < \varepsilon$
\end{theorem}

\quad

\begin{definition}
    $f$ функция на $[a, b]$, существование предела $S_{\tau, \theta}(f)$ при $\abs{\tau} \to 0 $ означает, что $f$ интегрируема по Риману и $lim = \int_a^{b} f dx$
\end{definition}

\quad 

\underline{\textit{Упражнение:}}  Доказать, что если у функции есть один разрыв первого рода, то она интегрируема по Риману


\begin{proof}

    По теореме Кантора $\forall \varepsilon > 0 \,\, \exists \, \delta > 0 : \forall x, y \in [a, b] : \abs{x - y} < \delta \Rightarrow \abs{f(x) - f(y)} < \frac{\varepsilon}{b - a}$

    Берем дробление $\tau : \abs{\tau}  < \delta $

    \[
        \abs{\sum_{j = 1}^n f(t_j) (x_j - x_{j - 1}) - \sum_{j = 1}^n \int_{x_{j - 1}}^{x_j} f(x) dx} = \abs{\sum_{j = 1}^n f(t_j) (x_j - x_{j - 1}) - \sum_{j = 1}^n f(c_j) \cdot (x_j - x_{j - 1}) dx} = 
    \]

    \[
        = \abs{
            \sum_{j = 1} ^ n (f(t_j) - f(c_j)) \cdot (x_j - x_{j - 1})
        } < \frac{\varepsilon}{b - a} \sum_{j = 1} ^ n (x_j - x_{j - 1}) = \varepsilon
    \]
\end{proof}


\underline{\textit{Задача:}}  $f$ имеет разрыв в точке $c$ первого рода и $f \in C[a, b), f \in C(c,b], \exists \lim_{x \to c-}{f}, \exists \lim_{x \to c+}{f}$, тогда $S_{\tau, \theta} f \to \int_a^c f + \int_c^b f$

\quad 

\begin{example}
    \begin{enumerate}
        \item \[
            \int_0^{a} e^x dx = \,? , \,\,\, \tau = \{ 0, \frac{a}{n}, \frac{2a}{n}, ..., \frac{na}{n} \}, \,\,\, \theta = \{ 0, \frac{a}{n}, ... , \frac{n - 1}{n} a \}
        \]
    
        \[
            S_{\tau, \theta} f = \sum_{j = 1}^n f \left( \frac{a_{j - 1}}{n} \right)(x_j - x_{j - 1}) = \frac{a}{n} \cdot \left( 1 + e ^ {a / n} + ... + e ^{\frac{n - 1}{n} a} \right) = \frac{a}{n} \cdot \frac{e ^ {\frac{na}{n}} - 1}{e ^ {a / n} - 1} \to e^a - 1
        \]
        \\
        \item
        \[
            1 ^ p + 2 ^ p + ... n ^ p = n ^ {p + 1} \cdot \left( \left( \frac{1}{n} \right) ^ p + ... + \left( \frac{n}{n} \right) ^ p \right)/ n \, \ceq
        \]

        \[
            \tau = \{ 0, \frac{1}{n}, ... , \frac{n}{n} \}, \theta = \{\frac{1}{n}, ... , \frac{n}{n}  \}, f(x) = x ^ p
        \]

        \[
            \ceq \, n ^ {p + 1} S_{\tau, \theta} f \Rightarrow \frac{1 ^ p + ... + n ^ p}{(n + 1) ^ p} \to \int_0^1 x ^ p dx = \frac{1}{p + 1} \Rightarrow 1 ^ p + ... + n ^ p \sim \frac{n ^ {p + 1}}{p + 1}
        \]

        \item интеграл Пуассона
        \[
            \int_0^\pi \log{(1 - 2a \cos{x} + a ^ 2)} dx
        \]

        \[
            f = \log{(1 - 2a \cos{x} + a ^ 2)},\, [0, \pi],\, \tau = \{0, \frac{\pi}{n}, ... , \frac{\pi n}{n} \}, \theta = \{0, \frac{\pi}{n}, ... , \frac{\pi (n - 1)}{n} \}
        \]

        \[
            S_{\tau, \theta} f = \sum_{j = 1}^n \frac{\pi}{n} \cdot \log{(1 - 2a \cos{\frac{\pi (j - 1)}{n} + a ^ 2})} = \frac{\pi}{n} \log{\prod_{j = 1}^n (1 - 2a \cos{\frac{\pi (j - 1)}{n} + a ^ 2})}
        \]

        Заметим, что $a ^ 2 - 2a \cos{\theta} + 1 = (a - e ^ {-i \theta})(a - e ^ {i \theta})$

        \[
            \prod_{j = 1}^n (1 - 2a \cos{\frac{\pi (j - 1)}{n} + a ^ 2}) =  \prod_{j = 1}^n (a - e ^ {\frac{-\pi (j - 1)}{n} i })(a - e ^ {\frac{\pi (j - 1)}{n} i }) = \frac{(a ^ {2n} - 1) \cdot (a - 1)}{(a + 1)} =
        \]

        \[
            =\frac{\pi}{n} \cdot \log{\frac{(a ^ {2n} - 1)(a - 1)}{a + 1}} \underset{n \to \infty}{\to} \begin{cases}
                0 & \abs{a} < 1 \\
                2 \pi \log{\abs{a}} & \abs{a} > 1 \\
                ? & \abs{a} = 1
              \end{cases}
        \]
    \end{enumerate}
    
\end{example}


\begin{remark}
    $\int_a^b f(x) dx \approx S_{\tau, \theta}(f)$
    \,\, Можно ли оценить погрешность?

    \[
        \abs{\int - S} \leqslant \sum \abs{f(t_j) - f(c_j)}(x_{j + 1} - x_j) \leqslant \sum \abs{t_j - c_j} \max{\abs{f'}}(x_{j + 1} - x_j) \leqslant M \cdot \abs{\tau} (b - a) \text{, где } \max{\abs{f'}} = M 
    \]
\end{remark} 
\quad 

\begin{remark}
    Более точная формула для приблеженного вычисления интеграла
\end{remark}



% Pattern Info
 
\tikzset{
pattern size/.store in=\mcSize, 
pattern size = 5pt,
pattern thickness/.store in=\mcThickness, 
pattern thickness = 0.3pt,
pattern radius/.store in=\mcRadius, 
pattern radius = 1pt}
\makeatletter
\pgfutil@ifundefined{pgf@pattern@name@_hmcbrwike}{
\pgfdeclarepatternformonly[\mcThickness,\mcSize]{_hmcbrwike}
{\pgfqpoint{0pt}{0pt}}
{\pgfpoint{\mcSize+\mcThickness}{\mcSize+\mcThickness}}
{\pgfpoint{\mcSize}{\mcSize}}
{
\pgfsetcolor{\tikz@pattern@color}
\pgfsetlinewidth{\mcThickness}
\pgfpathmoveto{\pgfqpoint{0pt}{0pt}}
\pgfpathlineto{\pgfpoint{\mcSize+\mcThickness}{\mcSize+\mcThickness}}
\pgfusepath{stroke}
}}
\makeatother
\tikzset{every picture/.style={line width=0.75pt}} %set default line width to 0.75pt        

\begin{tikzpicture}[x=0.75pt,y=0.75pt,yscale=-1,xscale=1]
%uncomment if require: \path (0,300); %set diagram left start at 0, and has height of 300

%Shape: Axis 2D [id:dp02029949369434092] 
\draw  (132,233.4) -- (534.5,233.4)(172.25,66) -- (172.25,252) (527.5,228.4) -- (534.5,233.4) -- (527.5,238.4) (167.25,73) -- (172.25,66) -- (177.25,73)  ;
%Straight Lines [id:da7710677350311679] 
\draw    (223,222) -- (223,242) ;
%Straight Lines [id:da6636852601941825] 
\draw    (290,223) -- (290,243) ;
%Straight Lines [id:da500250663548284] 
\draw    (267,224) -- (267,244) ;
%Straight Lines [id:da377281395464541] 
\draw    (243,222) -- (243,242) ;
%Straight Lines [id:da8386061762887231] 
\draw    (311,223) -- (311,243) ;
%Straight Lines [id:da7815801182452056] 
\draw    (333,224) -- (333,244) ;
%Straight Lines [id:da8616890669141692] 
\draw    (345,225) -- (345,245) ;
%Shape: Free Drawing [id:dp5560591048641171] 
\draw  [color={rgb, 255:red, 0; green, 0; blue, 0 }  ,draw opacity=1 ][line width=0.75] [line join = round][line cap = round] (222,195) .. controls (243.45,162.83) and (256.46,175.18) .. (301,172) .. controls (305.12,171.71) and (312.5,172.5) .. (317,171) .. controls (327.51,167.5) and (340.44,159.22) .. (343,149) .. controls (343.86,145.54) and (344,120.8) .. (344,128) ;
%Straight Lines [id:da9019023552875143] 
\draw [color={rgb, 255:red, 208; green, 2; blue, 27 }  ,draw opacity=1 ] [dash pattern={on 4.5pt off 4.5pt}]  (223,193) -- (223,222) ;
%Straight Lines [id:da3323747111843628] 
\draw [color={rgb, 255:red, 208; green, 2; blue, 27 }  ,draw opacity=1 ] [dash pattern={on 4.5pt off 4.5pt}]  (243,175) -- (243,192) -- (243,222) ;
%Straight Lines [id:da04332372509475069] 
\draw [color={rgb, 255:red, 208; green, 2; blue, 27 }  ,draw opacity=1 ] [dash pattern={on 4.5pt off 4.5pt}]  (267,174) -- (267,224) ;
%Straight Lines [id:da18598253754944805] 
\draw [color={rgb, 255:red, 208; green, 2; blue, 27 }  ,draw opacity=1 ] [dash pattern={on 4.5pt off 4.5pt}]  (290,173) -- (290,223) ;
%Straight Lines [id:da9229530047708664] 
\draw [color={rgb, 255:red, 208; green, 2; blue, 27 }  ,draw opacity=1 ] [dash pattern={on 4.5pt off 4.5pt}]  (311,173) -- (311,223) ;
%Straight Lines [id:da27194960638393784] 
\draw [color={rgb, 255:red, 208; green, 2; blue, 27 }  ,draw opacity=1 ] [dash pattern={on 4.5pt off 4.5pt}]  (332,163) -- (333,224) ;
%Straight Lines [id:da8974921338138321] 
\draw [color={rgb, 255:red, 208; green, 2; blue, 27 }  ,draw opacity=1 ] [dash pattern={on 4.5pt off 4.5pt}]  (344,125) -- (345,225) ;
%Straight Lines [id:da25565245177699003] 
\draw [color={rgb, 255:red, 74; green, 144; blue, 226 }  ,draw opacity=1 ]   (223,193) -- (243,175) ;
%Straight Lines [id:da5135245657261043] 
\draw [color={rgb, 255:red, 74; green, 144; blue, 226 }  ,draw opacity=1 ]   (243,175) -- (267,174) ;
%Straight Lines [id:da40438608893718364] 
\draw [color={rgb, 255:red, 74; green, 144; blue, 226 }  ,draw opacity=1 ]   (267,174) -- (290,173) ;
%Straight Lines [id:da8267126008581839] 
\draw [color={rgb, 255:red, 74; green, 144; blue, 226 }  ,draw opacity=1 ]   (290,173) -- (311,173) ;
%Straight Lines [id:da6583698282228275] 
\draw [color={rgb, 255:red, 74; green, 144; blue, 226 }  ,draw opacity=1 ]   (311,173) -- (332,163) ;
%Straight Lines [id:da6130580258949248] 
\draw [color={rgb, 255:red, 74; green, 144; blue, 226 }  ,draw opacity=1 ]   (332,163) -- (344,125) ;
%Straight Lines [id:da9281438699801965] 
\draw    (276,227) -- (276,244) ;
%Shape: Rectangle [id:dp8370752348674357] 
\draw  [pattern=_hmcbrwike,pattern size=8.100000000000001pt,pattern thickness=0.75pt,pattern radius=0pt, pattern color={rgb, 255:red, 126; green, 211; blue, 33}] (267,174) -- (290,174) -- (290,233) -- (267,233) -- cycle ;

% Text Node
\draw (252,245.4) node [anchor=north west][inner sep=0.75pt]    {$x_{j}$};
% Text Node
\draw (291,246.4) node [anchor=north west][inner sep=0.75pt]    {$x_{j+1}$};
% Text Node
\draw (272,248.4) node [anchor=north west][inner sep=0.75pt]    {$t_{j}$};


\end{tikzpicture}

Приближаем трапециями (формула трапеций)

\[
    \int_a^b f dx  \approx \sum (x_{j + 1} - x_j) \cdot \frac{f(x_j) + f(x_{j + 1})}{2}
\]

\begin{namedtheorem}{о погрешности в формуле трапеций}
$f \in C^2[a, b]$

\[
    \abs{\int_a^b f - \sum (x_{j + 1} - x_j) \cdot \frac{f(x_j) + f(x_{j + 1})}{2}} \leqslant \frac{1}{8} \abs{\tau} ^ 2 \int_a^b \abs{f''(x)}dx
\]
\end{namedtheorem}