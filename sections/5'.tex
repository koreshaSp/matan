\newpage
\section{Лекция *}


\subsection{Кривая}

\begin{definition}

    Путь в $R^m$ - это $\gamma : [a, b] \to R^m$ - непрерывное

    Носитель пути  : $\gamma([a, b]) \subset R^m$

    $\gamma(a)$ - начало, $\gamma(b)$ - конец

    Замкнутый путь, если $\gamma(a) = \gamma(b)$

    $\gamma$ - инъекция $\Rightarrow$ простой путь

    $\gamma(t) = (\gamma_1(t), \gamma_2(t), ... , \gamma_m(t))$

    $\gamma_j$ - координатная функция

    $\gamma $ - непр. $\Longleftrightarrow$ все $\gamma_j$ - непр


\end{definition}


\begin{definition}
    $\gamma_1: [a, b] \to R^m, \gamma_2 : [c, d] \to R^m$ - экв пути, если
    $\exists$ строго возр биекция $\phi : [a, b] \to [c, d]$, т.ч.
    $\gamma_2 \circ \phi = \gamma$
\end{definition}

\begin{remark}
    Это отношение эквивалентности
\end{remark}

\begin{definition}
    Кривая в $R ^ m$ - класс эквив. путей

    Представители --- параметризации
\end{definition}

Картинка 1

\begin{remark}
    Носитель одинаков для всех параметризаций внутри класса
\end{remark}


\begin{remark}
    Можно определить носитель кривой, начало, и конец
\end{remark}



\begin{definition}
    $\gamma$ - гладкий путь, если $\gamma_j \in C^1[a, b]$

    Гладкая кривая - кривая, у которой $\exists$ гладкая параметризация 
\end{definition}

Картинка 2

Картинка 3

\begin{definition}
   $\gamma : [a, b] \to R^m$ - путь $\theta$ = $\{t_0, ..., t_n \}$ --- дробление $[a, b]$, $a = t_0 < t_1 ... < t_n = b$

    $l_\theta(\gamma) = \Sigma_{j = 1}^n | \gamma(t_j) - \gamma(t_{j - 1})|$ - длина ломаной вписанной

    $ l(\gamma) = sup_\theta (l_\theta(\gamma))$
\end{definition}

\begin{remark}
    Длина есть у любого пути
\end{remark}

\underline{Вопрос:} Путь беск длины?

Кривая Пеано

$(x, x  ^ \alpha \cdot \sin{1 / x ^ \beta})$ - ?

\begin{theorem}
    $\gamma_1 \sim \gamma_2 \Rightarrow l(\gamma_1) = l(\gamma_2)$
\end{theorem}

\begin{proof}
    $\gamma_1 \circ \phi = \gamma$ $\theta$ - дробление $[a, b]$
    $\phi(\theta)$ - дробление $[c, d]$

    $$\Sigma |\gamma_1(t_j) - \gamma_1(t_{j - 1})| = \Sigma_{j = 1} ^ n |\gamma_2(\phi(t_j)) - \gamma_2(\phi(t_{j - 1}))|$$

    $$l_\theta(\gamma_1) = l_{\phi(\theta)}(\gamma_2)$$
\end{proof}

\begin{theorem}
    (аддитивность длины) $\gamma : [a, b] \to R^m, c \in (a, b)$

    $\gamma_- = \gamma |_{[a, c]}, \, \gamma_+ = \gamma |_{[c, b]}$
    $\Rightarrow l(\gamma) = l(\gamma_-) + l(\gamma_+)$

\end{theorem}

\begin{proof}
    $\theta_-$ - дробление $[a, c]$, $\theta_+$ - дробление $[c, b]$
     $\Rightarrow \theta = \theta_- \cup \theta_+$ - дробление $[a, b]$

    $l(\gamma) >= l_\theta(\gamma) = l_{\theta_-}(\gamma_-) + l_{\theta_+}(\gamma+)$

$\Rightarrow l(\gamma) >= sup_{\theta_-}(l_{\gamma_-}(\gamma_-)) + sup_{\theta_{\theta+}(l\theta_()\gamma+)} = l(\gamma_-) + l(\gamma_+)$

$\theta $ - дробление $[a, b] \to \theta_2 = \theta \cup {c} \to \theta_{\pm}$

$l_\theta(\gamma) <= l_{\theta_2}(\gamma) = l_{\theta-}(\gamma-) + l_{\theta+}(\gamma+) <= l(\gamma-) + l(\gamma+) \Rightarrow sup_\theta{l_{\theta}(\gamma)} <= l(\gamma-) + l(\gamma+)$

Картинка 4
\end{proof}

\begin{theorem}
    (о длинне гладкого пути)

    $\gamma : [a, b] \to R^m$, $\gamma_j\in C^1[a, b] \Rightarrow l(\gamma) = \int_a^b |\gamma'(t)| dt$, где
    $|\gamma'(t)| = \sqrt{\Sigma_{j = 1}^m |\gamma_j'(t)|^2}$
\end{theorem}


\begin{proof}
    $\theta $ - дробление $[a, b]$ $\theta : a = t_0 < t_1 < ... < t_n = b$

    $l_\theta(\gamma) = \Sigma_{j = 1}^n |\gamma(t_j) - \gamma(t_{j - 1})|$

    $$|\gamma(t_j) - \gamma(t_{j - 1})| = \sqrt{\Sigma_{k = 1}^ m |\gamma_k(t_j) - \gamma_k(t_{j - 1})| ^ 2} = $$ 
    
    $$ = \sqrt{\Sigma_{k = 1} ^ m |\gamma_k'(?)| ^ 2 \cdot (t_j - t_{j - 1}) ^ 2} = (t_j - t_{j - 1}) \sqrt{\Sigma_{k = 1} ^ m |\gamma_k'(?)|^2}$$

    $m_{j, k} = \min_{[t_{j-1}, t_j]} |\gamma_k'|$
    $M_{j, k} = \max{[t_{j-1}, t_j]} |\gamma_k'|$


    $(t_j - t_{j - 1}) \sqrt{\Sigma_{k = 1} ^ m m_{j, k}^2} \leqslant |\gamma(t_j) - \gamma(t_{j - 1})| \leqslant (t_j - t_{j - 1}) \sqrt{\Sigma_{k = 1} ^ m M_{j, k}^2}$

    $(t_j - t_{j - 1}) \sqrt{\Sigma_{k = 1} ^ m m_{j, k}^2} \leqslant \int_{t_{j - 1}}^{t_j} | \gamma'(t)| dt \leqslant (t_j - t_{j - 1}) \sqrt{\Sigma_{k = 1} ^ m M_{j, k}^2} $

    $\forall t \in [t_{j - 1}, t_j]$ $\sqrt{\Sigma_{k = 1} ^ m m_{j, k}^2} \leqslant |\gamma'(t)| \leqslant \sqrt{\Sigma_{k = 1} ^ m M_{j, k}^2} $

    Суммируем по j 

    л.ч. = $\Sigma (t_j - t_{j - 1}) \sqrt{?} \leqslant l_{\theta}(\gamma) \cup  \int_{a}^{b} | \gamma'(t)| dt \leqslant \Sigma (t_j - t_{j - 1}) \sqrt{?} = $ п.ч. 

    левое и правое стремятся к одному и то му же при $|\theta| \to 0$


    $\epsilon > 0$ Th. Кантора $\delta > 0 : \forall s, t \in [a, b] : |s - t| < \delta \forall k = 1...m |\gamma_k'(s) - \gamma_k'(t)| < \epsilon$

    $|M_{j, k} - m_{j, k}| < \epsilon$, если $|t_j - t_{j - 1}| < \delta$

    $\sqrt{\Sigma_{j = 1}^m M_{j, k}^ 2} - \sqrt{\Sigma_{j = 1}^m m_{j, k}^ 2} \leqslant \sqrt{\Sigma_{j = 1}^m (M_{j, k} - m_{j, k})^ 2} < \sqrt{m} \cdot \epsilon$

    |П.ч. - л.ч.| < $\Sigma (t_j - t_{j - 1}) \sqrt{m} \epsilon = (b - a) \sqrt{m} \epsilon$
    
    $|\theta| \to 0, |\theta| < \delta$

    Доказали: Л.ч. $(\theta) \leqslant l_\theta(\gamma) \cup \int_a^b \gamma'(t) dt \leqslant$ П.ч. $(\theta)$
\end{proof}

\underline{Следствия}

\begin{enumerate}
    \item (О длине графика функции)
    
    $f : [a, b] \to R, f \in C1[a, b], \Gamma_f = \{(x, f(x)) | x \in [a, b]^2\} \subset R^2$

    $l(\Gamma_f) = \int_a^{b}\sqrt{a + (f'(x))^2} dx$

    $(t, f(t)) \to^{d} (1, f'(t))$

    Картинка 5, 6

    \item $r(\phi), r : [\alpha, \beta] \to R+ r \in c'[\alpha, \beta]$
    
    Картинка 7

    $(x(t) = r(t)\cos{t}, y(t) = r(t)\sin{t})$

    $x'(t) = r'(t)\cos{t} - r(t) \sin{t}$

    $y'(t) = r'(t)\sin{t} + r(t)\cos{t}$

    $l = \int_\alpha^\beta \sqrt{(r')^2 + r^2} d\phi$
\end{enumerate}


\newpage
\section{Глава 5. Дифферинциальное исчесление функций \\ нескольких переменных}

$f : X \to Y$, $X, Y$ - линейные, нормированные + полнота, пространства над одним полем скаляров $(\mathbb{R}  / \mathbb{C})$


дифф в т. $x_0$ --- 
$f(x) = f(x_0) + A(x - x_0) + $ что-то малое $o(x - x_0)$

A --- должно быть линейным отображением

$f(x) = f(x_0) + A(x - x_0) + \alpha(x)$, где 
$\frac{||\alpha(x)||}{||x - x_0||} \to 0, x \to x_0 $


\subsection{Линейные и полелинейные непрерывные отображения}

\begin{definition}
    $X, Y$ - линейные пространства над одним полем скаляров, $U : X \to Y$ - линейное, если 

    (1) $U(x_1 + x_2) = U(x_1) + U(x_2)$

    (2) $U(\lambda x) = \lambda U(x)$
\end{definition}

\begin{remark}
    $U(x) = Ux$
\end{remark}

\begin{definition}
    $X_1, X_2, ..., X_n, Y$ $U : X_1 \times X_2 ... \times X_n \to Y$ - полелинейное, если оно линейно по каждому из аргументов
\end{definition}

\begin{example}
    \begin{enumerate}
        \item $X = C[-1, 1] \delta : X \to R, \delta(f) = f(0)$
        \item $X = C[a, b], Y = R, Uf = \int_a^b f dx$
        \item $X = C[a, b], Y = C[a, b], (Uf)(x) = \int_a^xf(t)dt$
        \item $X = C^1[a,b], Y = C[a, b], (Df)(x) = f'(x), D : X \to Y$
        \item $X_1 = X_2 = ... = R(C) = Y U(x_1, ..., x_n) = x_1 \cdot ... \cdot x_n$
        \item $X = R^m, X_2 = R^M, Y = R, U(x_1, x_2) = (x_1, x_2)$, комплексные нельзя, потому что не будет линейности по второй координате
        \item $X_1 = R ^ 3, X_2 = R ^ 3, Y = R ^ 3, U(x1, x2) = x_1 \times x_2 = [x_1, x_2]$
        \item $X_1 = ... = X_m = \R^m, Y = \R, U(x_1, ..., x_m) = det(x_1, ..., x_m)$
    \end{enumerate}
\end{example}


\begin{theorem}
    (о непрерывности линейного отображения)

    $U : X \to Y$ - линейное, X, Y - лин пространства над одним полем скаляров

    Следующие утверждения эквивалентны 

    \begin{enumerate}
        \item U - непр
        \item U - непр в 0
        \item $\exists C : \forall x \in X || U_x||_Y \leqslant C ||x||_X$
    \end{enumerate}
 \end{theorem}

 \begin{proof}
     (1) $\Rightarrow$ (2) - очевидно

     (2) $\Rightarrow$ (3) $\forall \epsilon > 0 \exists \delta > 0 : \forall x : ||x|| \leqslant \delta, ||U_x|| \leqslant \epsilon$

     $x \to x^\sim = x \cdot \frac{\delta}{||x||} \Rightarrow ||Ux^\sim|| \leqslant \epsilon = \frac{\delta}{||x||} \cdot ||U_x||$

     $C = \frac{\epsilon}{\delta}$ --- подходит

     (3) $\Rightarrow$ (1) --- липшецевость $||U(x_1) - U(x_2)|| \leqslant C ||x_1 - x_2|| \Rightarrow $ непрерывность

 \end{proof}

 \begin{theorem}
     $U : X_1 \times ... \times X_n \to Y$ --- полелинейное

     След. утверждения эквивалентны

     \begin{enumerate}
         \item U - непрерывное
         \item непрерывное в $0 = 0 \times ... \times 0$
         \item $\exists C : ||U(x_1, ... , x_n) || \leqslant C ||x_1|| \cdot ... \cdot ||x_n|| $
     \end{enumerate}

     $||(x_1, ... , x_n)||_{X_1 \times ... \times X_n} = ||x_1||_{X_1} + ... + ||x_n||_{X_n}$ (или какая-то другая)
 \end{theorem}

 \begin{definition}
     $U : X \to Y$ - линейное непрерывное отображением

     $||U|| = inf {\{ C | \forall x \in X,  ||Ux|| \leqslant C ||x||\} }$
 \end{definition}

 \begin{remark}
     inf достигается, т.е. $\forall x \in X, ||Ux||\leqslant ||U|| \cdot ||x||$
    
     В частности $||U|| \geqslant \frac{||Ux||}{||x||} $
 \end{remark}

 \begin{example}
     $U : C[a, b] \to C[a, b], (Uf)(x) = \int_a^xf(t)dt$

     $||Uf|| \leqslant (b - a) \cdot ||f||$

     $||U|| = b - a$
 \end{example}