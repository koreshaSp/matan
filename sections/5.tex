\section*{Лекция 5 (17.03.2022)}
\subsection{Вводная в кривые}
\begin{definition}{Путь}
    Путь в $\R^m x : [a, b] \mapsto \R^m$~--- непрерывное.
\end{definition}
\begin{definition}{Носитель пути}
    Носитель пути: $\gamma([a, b])\in \R^m$
\end{definition}
\begin{definition}
    Путь замкнут если $\gamma(a) = \gamma(b)$.
\end{definition}
\begin{definition}
    Если $\gamma$~--- инъекция, то такой путь называется
    простым.
\end{definition}
\begin{remark}
    Можно определять путь покоординатно:
    $\gamma(t) = (\gamma_1(t),\dots, \gamma_m(t))$ 
    Где $\gamma_i$~--- координатные функции.
\end{remark}
\begin{definition}
    Назовём два пути $\gamma_1\colon [a,b]\mapsto \R^m$ и
    $\gamma_2\colon [c,d]\mapsto \R^m$ эквивалентными,\\
    если $\exists \text{строго возрастающая биекция }
    \varphi\colon [a,b]\mapsto[c, d]$, так что 
    $\gamma_2\circ \varphi = \gamma_1$.
\end{definition}
\begin{remark}
    Это отношение эквивалентности.
\end{remark}
\begin{definition}
    Кривая в $\R_m$~--- класс эквивалентности путей
    Представители~--- параметризация.
\end{definition}
\begin{remark}
    На самом деле параметризаций у одного носителя может быть
    много. Например у полуокружности есть следующие параметризации:
    $(\cos x, \sin x), (x, \sqrt{1 - x^2})$
\end{remark}
\begin{remark}
    Зная параметризацию можно определить носитель кривой,
    начало и конец, но хочется понимать ещё про гладкость.
\end{remark}
\begin{definition}
    $\gamma$ ~--- гладкий путь, если $\gamma_i\in C_1[a,b]$.\\
    Гладкая кривая~--- кривая, у которой $\exists$ гладкая
    параметризация.
\end{definition}
\begin{remark}
    Нельзя утверждать, что у гладкой кривой есть явная касательная
    в каждой её точке.
\end{remark}
\subsection{Длина кривой}
\begin{remark}
    По факту, хотим подробить нашу кривую на маленькие отрезочки
    и засуммировать их, давайте подумаем как это сделать аккуратно.
\end{remark}
\begin{definition}
    $\gamma\colon [a, b]\mapsto \R^m$~--- путь.
    $\theta = \{t_0,\dots, t_n\}, a = t_0 < t_1 < \dots < t_n = b$~---
    разбиение $[a, b]$.
    $l_\theta(\gamma) = \sum\limits_{j = 1}^{n}
    \abs{\gamma(t_j) - \gamma(t_{j - 1})};\\$
    Тогда длиной назовём  $l(\gamma) = \sup_\theta l_\theta(\gamma)$
\end{definition}
\begin{remark}
    Заметим, что длина есть у любого пути, так как мы не вводили
    дополнительные требования на путь.
\end{remark}
\begin{remark}
    Посмотрим на $(x, x\sin \frac{1}{x})$
    Вопрос: Путь бесконечной длины?
\end{remark}
\begin{theorem}
    $\gamma_1 \sim \gamma_2\Rightarrow l(\gamma_1) = l(\gamma_2)$
\end{theorem}
% TODO: fix m;k macro for ~ symbol after math mode
\begin{proof}
    $\gamma_2\circ\varphi=\gamma_1$
    $\theta$~--- дробление $[a,b]$,
    $\varphi(\theta)$~--- дробление $[a,b]$.
    $\sum\limits_{j = 1}^{n}\abs{\gamma_1(t_j) - \gamma_1(t_{j-1})}=\\
    \sum\limits_{j=1}^{n}\abs{\gamma_2(\varphi(t_j))-\gamma_2(\varphi(t_{j-1}))}$.
    $l_{\theta}(\gamma_1) = l_{\varphi(\theta)}(\gamma_2)$
\end{proof}
\begin{theorem}
    $\gamma\colon [a,b]\mapsto\R^m, c\in(a,b)$.
    Пусть $\gamma_{-}=\gamma\big|_a^c, \gamma_{+} = \gamma\big|_c^b$.
    Требуется доказать, что $l(\varphi)=l(\varphi_-) + l(\varphi_+)$
\end{theorem}
\begin{proof}
    % TODO
    $\theta_-$~--- дробление $[a,c]$, $\theta_+$ ~-- дробление
    $[c,b]\Rightarrow \theta = \theta_- + \theta_+$~---
    дробление $[a,b]$
    Докажем неравенство в две стороны.
    \[
        l(\varphi)\geq l_{\theta}(\varphi) = l_{\theta_-}(\varphi_i) + l_{\theta_+}(\varphi_i)\Rightarrow\\
    \]\[
        l(\varphi)\ge \sup_{\theta_-}l_{\theta_-}(\gamma_-) + 
        \sup_{\theta_+}l_{\theta_+}(\gamma_+) = l(\gamma_-) + l(\gamma_+)
    .\] 
    Докажем в обратную сторону.
    $\theta$ ~--- дробление $[a,b]\rightarrow 
    \overline{\theta} = \theta \cup \{c\}\rightarrow \theta_{+/-}$.
    \[
        l_{\theta}(\gamma)\leq l_{\overline{\theta}}(\gamma) =
        l_{\theta-}(\gamma_-) + l_{\theta+}(\gamma_+)\leq
        l(\gamma_i)+l(\gamma_+)\Rightarrow 
        \sup_{\theta}l_{\theta}(\gamma)\leq l(\gamma_-) + l(\gamma_+)
    .\] 
\end{proof}
\begin{namedtheorem}{O длине гладкого пути}
    $\gamma\colon[a,b]\mapsto\R^m, \gamma_j=C'[a,b] \Rightarrow 
    l(\gamma)=\bigintss_{a}^{b}\abs{\gamma'(t)}dt$, где
    $\abs{\gamma'(t)} = \sqrt{\sum\limits_{j=1}^{m}{\abs{\gamma_j(t)}}^2}$
\end{namedtheorem}
\begin{proof}
    $\theta$ ~--- дробление $[a,b]$,  
    $\theta\colon a = t_0<\dots<t_n=b$ 
    $$l_{\theta}(\gamma)=\sum\limits_{j=1}^{m}{\gamma(t_j) - \gamma(t_{j-1})}$$
     \[
         \abs{\gamma(t_j)-\gamma(t_{j-1})}=
         \sqrt{\sum\limits_{k=1}^{m}\abs{\gamma_k(t_j)-\gamma_k(t_{j-1})}^2}=
         \sqrt{\sum\limits_{k=1}^{m}{\abs{\gamma'_k(?)}^2(t_j-t_{j-1})^2}}=
         (t_j - t_{j-1})\sqrt{\sum\limits_{k=1}^{m}{\abs{\gamma'_k(?)}^2}}
     .\] 
     Пусть $m_{j,k}=\min\limits_{[t_{j-1},t_j]} \abs{\gamma_k'}$,
     $M_{j,k}=\max\limits_{[t_{j-1},t_j]}\abs{\gamma'_k}$.
     Тогда можно продолжить  следующим образом:
     $$(t_j - t_{j-1})\sqrt{\sum\limits_{k=1}^{m}{m_{j,k}^2}}\leq
     \abs{\gamma(t_j) - \gamma(t_{j-1})}\leq
     (t_j - t_{j-1})\sqrt{\sum\limits_{k=1}^{m}{M_{j,k}^2}}$$
     С другой стороны:
     $$(t_j - t_{j-1})\sqrt{\sum\limits_{k=1}^{m}{m_{j,k}^2}}\leq
     \int\limits_{t_{j-1}}^{t_j}{\abs{\gamma'(t)}dt}\leq
     (t_j - t_{j-1})\sqrt{\sum\limits_{k=1}^{m}{M_{j,k}^2}}$$
     Суммируем по $j$:
     \[
         \sum\limits_{}^{}{(t_j-t_{j-1})\sqrt{?}}\leq
         \begin{cases}
             l_{\theta}(\gamma)\\
             \int\limits_{a}^{b}{\abs{\gamma'(t)}dt}
         \end{cases}\leq
         \sum\limits_{}^{}{(t_j-t_{j-1})\sqrt{?}}
     .\] 
     Заметим, что обе части стремятся к одному и тоже при
    $\abs{\theta}\rightarrow 0$.

     Тогда для  $\varepsilon > 0$ по теореме кантора $\delta > 0\colon\\
     \forall s,t\in [a,b]\colon \abs{s - t} < \delta, 
     \forall k = 1\dots m, \abs{\gamma'_k(s)-\gamma'_k(t)} < \varepsilon$ 
     Тогда $\abs{M_{j,k}-m_{j,k}}<\varepsilon$ если $t_j - t_{j-1}<\delta$.
     \[
         \sqrt{\sum\limits_{k=1}^{m}{M^2_{j,k}}}-
         \sqrt{\sum\limits_{k=1}^{m}{m^2_{j,k}}} \leq
         \sqrt{\sum\limits_{k=1}^{m}{(M_{j,k} - m_{j,k})^2}}<
         \sqrt{m}\varepsilon
     .\] 
     Тогда $$\abs{\text{п.ч - л.ч}} <
     \sum\limits_{k=1}^{m}{(t_j-t_{j-1})\sqrt{m}\varepsilon}=
     (b - a)\sqrt{m}\varepsilon$$
\end{proof}
\begin{follow}
    \begin{enumerate}
        \item (Длина графика функции)\\
            Пусть $f\colon[a,b]\mapsto \R, f\in C^1[a,b], 
            G_f=\{(x,f(x)) \mid x\in [a,b]\}\subseteq \R^2$.
            Тогда длина графика функции равна:
            $l(G_f)=\bigintss\limits_{a}^{b}{\sqrt{1 + (f'(x))^2}dx}$
        \item
            Есть функция 
            $r\colon [\alpha, \beta]\mapsto \R_+, r\in C^1[\alpha, \beta]$
            Тогда можно запараметризовать эту кривую как:
            $(r(t)\cos t, r(t)\sin t)$, обозначим за координаты
            $x(t), y(t)$ соответственно.

            Тогда:
             $x'(t) = r'(t)\cos t - r(t)\sin t; 
             y'(t) = r'(t)\sin t + r(t)\cos t$

             Значит длину кривой можно вычислить следующим образом:
             $l = \bigintss\limits_{\alpha}^{\beta}{\sqrt{(r')^2 + r^2}d\varphi}$
    \end{enumerate}
\end{follow}
\section{Дифференциальное исчисление функций нескольких переменных}
\begin{remark}
    Прежде чем начать, подумаем что нам вообще надо.
    Производную можно представлять как линейное приближение
    функции в какой-либо точке.
    $f\colon X\mapsto Y$, где $X,Y$~--- линейные, нормированные,
    полные, пространства над одним полем
    скаляров($\R$ или $\mathbb{C}$).
    $f(x) = f(x_0) + A(x-x_0) + o(x - x_0)$.
    Объясним почему наложены те или иные условия.
    В одномерье $A$ было числом. В общем случае $A$ должно
    быть линейным отображением, поэтому нам хочется чтобы поле
    скаляров у $X,Y$ было одним и тем же.
    Более того, хочется иметь возможность переходить к пределу,
    а когда мы переходим к пределу естественно хотеть иметь
    корректное расстояние, поэтому нам нужна нормированность и полнота.
\end{remark}
\subsection{Линейные отобрежения}
\begin{definition}
    $X, Y$ ~--- линейные пространства над одним полем скаляров.
    $U\colon X\mapsto Y$ ~--- линейное, если
    \begin{enumerate}
        \item (Аддитивность)\\
            $U(x_1 + x_2)=U(x_1)+U(x_2)$
        \item (Однородность)\\
            $U(\lambda x) = \lambda U(x)$
    \end{enumerate}
\end{definition}
\begin{definition}
    Пусть $X_1, X_2, \dots, X_n, Y$ ~--- пространства над одним 
    полем скаляров.
    Тогда $U\colon X_1\times\dots\times X_n\mapsto Y$~--- полилинейное,
    если оно линейно по каждому из аргументов.
\end{definition}
\begin{remark}
    Часто скобочки опускаются: $U(x) = Ux$
\end{remark}
\begin{example}
    \begin{enumerate}
        \item
            $X = C[-1, 1], \delta\colon X\mapsto \R, \delta(f)=f(0)$. 
            Тогда $\delta$~--- линейное отображение.
        \item
        $X = C[a,b], Y = \R$. Тогда 
        $Uf = \bigintss\limits_{a}^{b}{fdx}$ ~---
        линейное отображение.
        \item
            $X = C[a,b], Y = C[a,b]$. Тогда
           $(Uf)(x) = \bigintss\limits_{a}^{x}{f(t)dt}$ 
        \item
            $X = C^1[a,b], Y = C[a,b]$. Тогда 
            $(Df)(x) = f'(x), D\colon X\mapsto Y$ тоже
            линейное отображение.
        \item
            $X_1 = X_2 = \dots X_n = \R = Y$. Тогда
            $U(x_1,\dots, x_n) = x_1\cdot \dots \cdot x_n$~---
            полилинейное отображение.
        \item
           $X_1 = \R^m, X_2 = \R^m, Y = \R$. Тогда
           $U(X_1, X_2) = (X_1, X_2)$, где $(X_1, X_2)$~---
           скалярное произведение, линейное по первой координате.
        \item
            $X_1 = \R^3, X_2 = \R^3, Y = \R^3$. Тогда
            $U(x_1, x_2) = x_1\times x_2 = [x_1, x_2]$, где
            $[x_1,x_2]$~--- линейное отображение,
            полилинейное отображение.
        \item
            $X_1 = \dots = X_m = \R^m$. Тогда 
            $U(x_1,\dots, x_m) = \det(x_1, \dots, x_m)$ 
            полилинейное отображение.
    \end{enumerate}
\end{example}
\begin{namedtheorem}{О непрерывности линейного отображения}
    $U\colon X\mapsto Y$ ~--- линейное, $X,Y$ ~--- линейные,
    нормированные отображения (далее лно)
    над одним полем скаляров.
    Тогда следующие утверждения эквивалентны:
    \begin{enumerate}
        \item$U$ ~--- непрерывно
        \item $U$ ~--- непрерывно в 0
        \item $\exists C\colon \forall x\in X, \abs{\abs{Ux}}_Y\leq C\abs{\abs{x}}_x$
    \end{enumerate}
\end{namedtheorem}
\begin{proof}
    \begin{enumerate}
        \item $(1)\Rightarrow (2)$~--- очевидно
        \item  $(2)\Rightarrow (3)$
             $\forall \varepsilon > 0, \exists \delta > 0\colon
             \forall x\colon \norm{x}\leq\delta, \norm{Ux}\leq \varepsilon$.
             $x \rightarrow \overline{x} = x\frac{\delta}{\norm{x}}
             \Rightarrow \norm{U\overline{x}}\leq \varepsilon$ 
             % TODO недоделано
        \item $(3)\Rightarrow(1)$
             % $\|U(x_1) - U(x_2)\| \leq C\|x_1 - x_2\|$~---
             липшецевость  $\Rightarrow$ непрерывное.
    \end{enumerate}
\end{proof}
\begin{theorem}
    $U\colon X_1\times\dots \times X_n\mapsto Y$~--- полилинейное.
    Тогда следующие утверждения эквивалентны:
    \begin{enumerate}
        \item
            $U$ ~--- непрерывно
        \item 
            $U$ ~--- непрерывно в 0
        \item
            $\exists C\colon \norm{U(X_1,\dots, X_n)} \leq
            C\norm{x_1}\cdot\ldots\cdot\norm{x_n}$
    \end{enumerate}
\end{theorem}
\begin{remark}
    % TODO
\end{remark}
\begin{definition}
    $U: X\rightarrow Y$ ~--- лно.
    $\norm{U} = \inf\{C \mid \forall x\in X \norm{Ux}\leq C\norm{x}\}$
\end{definition}
\begin{remark}
$\inf$ достигается то есть: $\forall x\in X, \norm{Ux}\leq
    \norm{U}\cdot \norm{x}$
\end{remark}
\begin{example}
    $U\colon C[a,b]\mapsto C[a,b]$, причём
    $(Uf)(x) = \bigintss\limits_{a}^{x}{f(t)dt}$.
    Хотим оценить.
    % TODO
\end{example}
